%--------------------------------------------------------------------%
%
% Berkas utama templat LaTeX.
%
% @author Petra Barus, Peb Ruswono Aryan
% updated by Dionesius Agung (2020)
%--------------------------------------------------------------------%
%
% Berkas ini berisi struktur utama dokumen LaTeX yang akan dibuat.
%
%--------------------------------------------------------------------%

\documentclass[a4paper, 12pt, oneside, onecolumn, final, openany]{report}

\input{utils/if-itb-thesis.sty}

\makeatletter

\makeatother

\bibliography{references}

\title{SECURING HOME NETWORK BY UTILIZING FREE AND OPEN SOURCE ENTERPRISE-GRADE NETWORK FIREWALL HOSTED ON PERSONAL COMPUTER}
\date{June 2023}
\author{
	Afiq Nazrie Rabbani
}

\usepackage{subfiles}

\begin{document}

% Basic configuration
\pagenumbering{roman}
\setcounter{page}{0}

\subfile{chapters/cover}
\subfile{chapters/copyright}

% ganti menjadi approval-single-advisor jika pembimbing 1 orang
% \subfile{chapters/approval-multiple-advisors}
\subfile{chapters/approval-single-advisor}

\subfile{chapters/title}

% \subfile{chapters/statement}

\pagestyle{plain}

% Frontmatter

\subfile{chapters/declaration}
\subfile{chapters/dedication}
\subfile{chapters/acknowledgement}

\subfile{chapters/abstract-en}
\subfile{chapters/abstract-my}

% \subfile{chapters/forewords}

% Hacks to capitalize all chapter-level titles in ToC
\renewcommand*\contentsname{TABLE OF CONTENTS}
\renewcommand*\appendixtocname{LIST OF APPENDICES}
\renewcommand*\listfigurename{LIST OF FIGURES}
\renewcommand*\listtablename{LIST OF TABLES}
\renewcommand*\bibname{REFERENCES}

% Lanjutan frontmatter
\tableofcontents
\listofappendices
{%
	\let\oldnumberline\numberline%
	\renewcommand{\numberline}{\figurename~\oldnumberline}%
	\listoffigures%
}
{%
	\let\oldnumberline\numberline%
	\renewcommand{\numberline}{\tablename~\oldnumberline}%
	\listoftables%
}


%----------------------------------------------------------------%
% Konfigurasi Bab
%----------------------------------------------------------------%
%------------------------------------------------------%
% Hack: 2 baris berikut dipindah ke chapter-1.tex
%   previous method: nomor halaman sebelum BAB I jadi 0
%------------------------------------------------------%
% \pagenumbering{arabic}
% \setcounter{page}{0}
\renewcommand{\chaptername}{CHAPTER}
% \renewcommand{\thechapter}{\Roman{chapter}}
%----------------------------------------------------------------%

%----------------------------------------------------------------%
% Dafter Bab
% Untuk menambahkan daftar bab, buat berkas bab misalnya `chapter-6` di direktori `chapters`, dan masukkan ke sini.
%----------------------------------------------------------------%
\subfile{chapters/chapter-1}
\subfile{chapters/chapter-2}
\subfile{chapters/chapter-3}
\subfile{chapters/chapter-4}
\subfile{chapters/chapter-5}
\subfile{chapters/chapter-6}
%----------------------------------------------------------------%

% Daftar pustaka
\begingroup
\renewcommand{\baselinestretch}{1.0}
\printbibliography[heading=bibintoc]
\endgroup

% Before:
% ---
% Index
% \appendix
% \addcontentsline{toc}{part}{Lampiran}
% \part*{Lampiran}
% ---

% Format judul bab lampiran
\titleformat{\chapter}[hang]
{\large\bfseries}
{\chaptertitlename\ \thechapter}{1em}
{\large\bfseries}
\titlespacing*{\chapter}{0pt}{-1.5\baselineskip}{\parskip}

\begin{appendices}
	\subfile{chapters/appendix-1}
	% \subfile{chapters/appendix-2}
\end{appendices}

\end{document}
