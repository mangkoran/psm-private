\documentclass[../index.tex]{subfiles}

\begin{document}

\chapter{SYSTEM DEVELOPMENT METHODOLOGY}

This chapter defines the development methodology which acts as a guideline to fulfil project's aims
and objectives. The selected methodology will be justified based on how the methodology will help
the development of the project. It will be followed by the explanation of the phases of the selected
methodology. Then, the system requirements will be analysed in order to be able to develop a
reliable system. The chapter will be ended by the conclusion of the previous discussion regarding
the selected methodology, development phases based on the selected methodology, and system
requirement analysis for the project development.

\subsection{Methodology Selection and Justification}

The primary objective of this project is to successfully develop a hardware firewall which small
business can use to secure their local network. As this project will only involves integration and
configuration of existing software and technology without any application development, the waterfall
model is selected as the development methodology. Waterfall model is selected as the developed
system appliance mainly serve its purpose or function without user-facing interface. This is in
accordance with the definition of waterfall model where the product definition is required to be
stable among the course of the development. Another reason is the project involves hardware
development or configuration where manufacturing or procurement cost are involved. In addition, the
developed system appliance could be considered as critical system which requires extensive safety
and security analysis of the solution specification and design (Balaji \& Murugaiyan, 2012).

\subsection{Phases of Selected Methodology}

There are five phases which ...

\end{document}
