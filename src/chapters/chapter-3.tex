\documentclass[../index.tex]{subfiles}

\begin{document}

\chapter{SYSTEM DEVELOPMENT METHODOLOGY}

This chapter defines the development methodology which acts as a guideline to fulfil project's aims
and objectives. The selected methodology will be justified based on how the methodology will help
the development of the project. It will be followed by the explanation of the phases of the selected
methodology. Then, the system requirements will be analysed in order to be able to develop a
reliable system. The chapter will be ended by the conclusion of the previous discussion regarding
the selected methodology, development phases based on the selected methodology, and system
requirement analysis for the project development.

\section{Methodology Selection and Justification}

The primary objective of this project is to successfully develop a hardware firewall which small
business can use to secure their local network. As this project will only involves integration and
configuration of existing software and technology without any application development, the waterfall
model is selected as the development methodology. Waterfall model is selected as the developed
system appliance mainly serve its purpose or function without user-facing interface. This is in
accordance with the definition of waterfall model where the product definition is required to be
stable among the course of the development. Another reason is the project involves hardware
development or configuration where manufacturing or procurement cost are involved. In addition, the
developed system appliance could be considered as critical system which requires extensive safety
and security analysis of the solution specification and design (Balaji \& Murugaiyan, 2012).

\section{Phases of Selected Methodology}

There are five phases which comprises the Waterfall methodology.

\subsection{Requirements Analysis and Definition}

The Waterfall project methodology starts by defining the project requirements. The system
services, constraints, and objectives are established with the system users. They are then defined
in detail and serve as a system specification.

For this system development project, this phase will discuss several network security concerns that
exist in home network environment. After the issues and concerns have been identified, the possible
solution or countermeasures will be discussed. The outcome of the discussion for this phase should
be used as the reference for considerations and justifications in later phases.

\subsection{System Design}

Following the requirements analysis is the system and software design. The
system design process defines the required condition to either software or physical
hardware systems. It determines the entire system architecture in general view.

In this phase, the overall firewall system design will be defined based on the
result of requirement analysis. Both of the hardware and software implementation will
be discussed and planned to be applied in later system development phases

\subsection{Implementation}

The implementation is where the design is accomplished as a set of solution or
solution units. This also includes verification that every implemented unit fulfill its
requirements.

When the system development project has reached the Implementation phase,
any software and hardware solutions based on the outcome of Requirement Analysis
and System Design phases will be installed and configured. Initially, the system
hardware will be assembled in accordance to the hardware available or selected by the
user. After the hardware have been assembled, the software can start to be installed
and configured based on the requirements


\subsection{Integration and System Testing}

In the integration and system testing phase, the particular solution units or
solutions are implemented and examined as an entire system to guarantee that the
project requirements have been fulfilled. After the system has been thoroughly tested,
it will be released to client.

In Integration and System Testing phase, the developed system will be tested
based on the defined testing methodology. This phase will ensure that both the
hardware and software aspect of the system is running properly and provide security
to the home network according to the system design.


\subsection{Operation and Maintenance}

After the initial system development, configuration, and testing is done and passed, the system will
be utilized based on the use case. The maintenance stage includes amending the system if in the
later stage there are unintended behaviors that were not found in previous system development phase.

Maintenance involves correcting errors that were not discovered in earlier
stages of the life cycle, improving the implementation of system units, and enhancing
the system services as new requirements are discovered.
After the device is deployed and put into use in actual home network, it will be
monitored to analyze its usage and effectiveness in providing network security inside
the household.

\section{Technology Used}

To achieve the project objectives defined in previous chapter, the following
technology will be used:

\subsection{Proxmox VE}

Proxmox VE is a type 1 hypervisor which serves as server management
platform for enterprise virtualization. It operates directly on a bare-metal system by
utilizing the Linux Containers (LXC) and KVM hypervisor as well as software-defined
networking and storage, all of that implemented on a single platform.

Proxmox will be used as the appliance's base system. Afterwards, the firewall OS and Pi-hole
container will be deployed on top of it.

\subsection{OPNSense}

OPNsense is an open-source firewall and routing platform based on FreeBSD.
A few of its features including traffic shaping, forward caching proxy, OpenVPN client
and intrusion detection system.

OPNsense will be used as the appliance's main OS. It will provide the network- based firewall and
network routing for the local network. A necessary configuration is required to fulfill the
project's objective.

\subsection{LXC}

LXC is one of type 1 hypervisor technology used to run multiple separate
Linux system in the form of container on a control host which utilize single Linux
kernel.

LXC is selected to be used for the project as it has minimal overhead compared
to other deployment method. Instead of creating an entire virtual machine, LXC
achieves its virtualization by utilizing virtual environment which has separate network
and process space.

In the system development project, LXC will be use for the deployment of Pi-
hole application. Instead of virtualizing the entire operating system only for Pi-hole,
LXC can be utilized to provide the necessary virtualization with noticably lower
overhead.

\subsection{Pi-hole}

Pi-hole provides a DNS sinkhole functionality which protects devices
connected to the local network from unwanted content, such as advertisement and
internet tracker, without being required to install any client-side software.

In this project, Pi-hole will be implemented in the system as an advertisement
and internet tracker blocker. Any DNS query will be sent to Pi-hole and Pi-hole will
determine if the DNS queries should be forwarded or allowed or it should be blocked
based on the applied rules.

\section{System Requirement Analysis}

To ensure that the developed appliance met the project objectives and deliver
its functionality as optimal as possible, there are several requirements that are required.

\subsection{Hardware Requirement}

The main objective of this project is to develop a low-cost hardware firewall
by utilizing an old PC. As PC is becoming more accessible to every person, it is chosen
as the base hardware which the firewall and additional application will be deployed.

However, to make sure that the appliance will serve its purpose optimally, several specific
requirements are needed.

The main objective of this project is to configure a network-based firewall
hosted on a personal computer (PC). As this project make use of Proxmox and LXC
for the virtualization, a decent PC with multi-cores CPU and high-speed disk are
required as to make the system responsive and capable to fulfill tasks

\subsection{Software Requirement}

For accessing the system through command line interface (CLI), a PC with
secure shell (ssh) capability is required. In Linux environment, the openssh package
requires the following dependencies:

To access the system through web interface, a modern web browser is required.
This includes any recent release of Firefox, Chromium-based browsers, and Safari

\section{Summary}

This chapter discussed the detailed steps for the implementation or
developments of the proposed system. Technologies and system requirement which
covers both hardware and software requirements are defined further. Clearly defining
the methodology and system requirements will help to ensure the project is completed
and its objectives and deliverables are achieved within the allocated period. The
system requirement, both software and hardware, also discussed for the purpose of
keeping the system running optimally. This chapter discussed the detailed steps for the
implementation or developments of the proposed system. Technologies and system
requirement which covers both hardware and software requirements are defined
further. Clearly defining the methodology and system requirements will help to ensure
the project is completed and its objectives and deliverables are achieved within the
allocated period. The system requirement, both software and hardware, also discussed
for the purpose of keeping the system running optimally

\end{document}
