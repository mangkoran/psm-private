\documentclass[../index.tex]{subfiles}

\begin{document}

\chapter{REQUIREMENT ANALYSIS AND DESIGN}

This chapter examines the proposed project requirements and system design. Identifying project
requirements and defining the proposed system design are mandatory to successfully deliver the
project objectives and functionalities.

\section{Requirement Analysis}

Unified Modeling Language (UML) is chosen as a method to analyze the project requirements. UML is
a graphically based notation, which is developed by the Object Management Group as a standard means
of describing software-oriented designs. It contains several different types of diagrams, which
allow different aspects and properties of a system design to be implemented.

However, for this particular project, only a few of UML diagrams will be used. This due to the
project development does not involve any creation of new software or interface. Rather, this project
implements off-shelf software solutions in a defined hardware.

The functional requirements are described using UML use case diagram and activity diagram. The
non-functional requirement will review the performance, usability, availability, reliability, and
security aspect of the system.

\subsection{Functional Requirement}

\subsubsection{Use Case Diagram}

Use case diagram is a visual depiction of a user's possible interactions with a certain system.
A use case diagram defines varying use cases and different types of users the system has and
commonly delivered with other types of diagrams as well. Use case diagram offers thorough summary of
the entire system with simple illustration. Use case diagram is also considered as the perfect
method to provide the overall view of the system at the early stage (Shen & Liu, 2003).

The use case diagram below was designed based on the result of the requirement analysis. In the
developed system, there exist three actors which are the user, the system, and the administrator.
Four use cases were also defined, which are accessing local network, accessing wide area network,
configuring the firewall, and monitoring the system. In addition, it followed by a table which list
the brief use case description.

\end{document}
