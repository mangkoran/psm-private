\documentclass[../index.tex]{subfiles}

\begin{document}

\chapter{CONCLUSION}

This chapter will discuss and summarize the achievements and findings gathered from each of project
chapter. Moreover, this chapter will address the advantages and disadvantages of the developed
system. Suggestion for future improvement will be also discussed at the end of the chapter.

\section{Achievement of Project Objectives}

Chapter 1 identified the current issue regarding local area network vulnerability in home
environement. As the guidance for this project, this chapter defined the objectives that should be
fulfilled and project scopes to limit the project development to the defined extent. The project
report organization is also discussed to help with the report writing.

Chapter 2 discussed regarding existing system analysis which focused on examining the current state of
available systems. This phase helped to understand how the existing systems work and identify any
noticable shortcomings or areas that require improvement.

Chapter 3 reviewed the methodology to be used to develop the system. Following the decision of the
methodology to be used, each of the selected methodolgy phase is identified and elaborated. System
requirement which comprise of hardware and software requirement is also addressed, including the
role of each requirement in the system development.

Chapter 4 ...

Chapter 5 ...

Chapter 6 ...

\section{System Advantages}

This system is designed to be an improvement to existing systems. One of the advantages of this
system is the utilization of hypervisor and virtualized environment as the system deployment model.
This approach allows multiple guest virtual machines and containers to be deployed in a single
hardware, which is a cost-saving measure for home use.

The components used were selected accordingly to maximize the functionality given with the limited
resource. OPNSense as the primary firewall OS provides user with the ability to configure firewall
rules and utilize Suricata intrusion detection system. The system also employs Pi-hole as DNS server
with DNS sinkhole capability that could filter unwanted content such as advertisement and telemetry.
With these capabilities, Pi-hole could help the system and network to reclaim resource such as
computing power and bandwith from wasted usage. To monitor all of these subsystems metrics and
behaviors, an LXC container is deployed with the proposed monitoring stack. This monitoring stack
will process metrics and logs of all deployed subsystems and curate it as a centralized dashboard
with essential information.

\section{Suggestion for Future Improvement}

The only faultless system is a system that is never developed. Based on the outcome of the developed
system, there are several suggestions and improvements that could be applied to enhance the system
on fulfilling the requirements further.

Since the beginning of 2023, there are several new mini PC released with decent specification that
could be a good alternative as hardware of this project. One of these devices is Topton i3 N305
fanless mini PC. This mini PC utilizes low power Intel i3 N305 processor, and 8 cores 8 threads CPU
with only 15W TDP. The use of low power CPU could help on saving electric bill in long term usage.
This mini PC also has 4 Intel i226-V 2.5G RJ45 ports, making it suitable to be used as networking
appliance like this project. Furthermore, the small dimension of this mini PC allows for easy
management and placement.

The firewall usage could also be improved. OPNSense has multiple additional network security plugins
that are available to be installed, such as ZenArmor and CrowdSec. These softwares have its own
advantages and disadvantages depending on the use case. In future improvement, these softwares could
be examined and compared to be used with this system as an alternative to the current
implementation.

\end{document}
