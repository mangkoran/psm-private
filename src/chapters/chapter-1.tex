\documentclass[../index.tex]{subfiles}

\begin{document}

\chapter{INTRODUCTION}
% Hack: gatau kenapa harus gini
\pagenumbering{arabic} \setcounter{page}{1}

Information and communication technology, in the shape of physical and digital assets, are uniquely
new kinds of personal property in this digital age. Home networks, every device connected to it,
including personal computers and smart devices, and even the data stored inside previously mentioned
devices should be treated carefully and protected under an adequate network security measure.
Personal data and even devices access control are two of many private assets that individual might
not want them to be accessed or even tampered by unknown outsiders. Internet Service Provider (ISP)
on most of the time provide their home internet subscriber a basic router with Wi-Fi capability.
However, the supplied router often has limited capabilities for securing the home network. Home
users can always purchase third party router with more features; however, it will still cost them a
fair amount of money.

This project is aimed to provide proof-of-concept in developing home network security by utilizing
enterprise-grade firewall deployed on virtualised environment on a personal computer (PC). The
firewall is able to be configured according to the use case of the user, with additional advanced
features and controls which can be found in enterprise firewall appliance. The usage of PC in this
project is intended to make the firewall deployment flexible; any PC that fulfil the minimum
requirements should be able to be used as the host of the firewall system.

\section{Problem Background}

For the past few years, people have seen the rapid growth of information and communication
technology, both the development and application in their everyday life. Since the world becoming
more interconnected, the adoption of technology continues to be among the deciding factors in the
development of modern human civilization. A study conducted by Pew Research shows that among
American households in 2018, 92 percent of them had one type of computer at the minimum and 85
percent of them were subscribed to a broadband Internet service. The personal computer and the
Internet have becoming two important technologies in the daily personal and work life of most
people. Both information technologies allow humans to socialize with each other, continue their work
from home or other places, and fulfil their entertainment needs. Furthermore, the emergence of
Industry Revolution 4.0 introduces various breakthrough in the applications of information and
communication technology. Among the new innovations is the Internet-of-Things (IoT), an idea of
computer science which is used to make regular devices "intelligent". IoT enables the development of
smart home appliances, such as doorbells that capture real-time footage of the front door,
refrigerators that notify its owner whenever their vegetables run out, and televisions that play TV
shows and movies that requested by their owners using voice command. It is forecasted that 21
percent of households worldwide will become smart homes by 2025. This increase will be stimulated by
smart lighting and controls and consumer electronics, with the latter contributing up to 40 percent
of the total IoT products in smart home by 2021.

Besides supporting human in their daily life routine, the widespread usage of information and
communication technology in households also introduces another type of risks and threats towards ICT
assets, both physical and digital form. Additionally, individuals in the home network environment
are far less capable and prepared to address cyberattacks and intruders. Because of that,
individuals at home are clueless to some extent as they may have no idea what is or may happen to
their home, home devices, the data stored within the devices, and more terrifying, their own
personal lives as attackers are now able to do physical harm to individuals inside the home.
Moreover, as detailed by Zetter, it is possible for hackers to perform attack against home or even
business automation and security systems through power lines. Using this conduit, the hackers can,
through remote commands, take control of a multitude of devices, such as lights, electronic locks,
heating and air conditioning systems, and security alarm and cameras. This exploit is possible as
the systems operate on Ethernet networks that communicate over the existing power lines in the home.

To safeguard the home network, one of the possible actions that homeowner and home LAN user can take
is to implement a network firewall. They need a dedicated network firewall in order to achieve
trustworthy protection for private network. Most of the time, home network subscribers are provided
a router by their ISP. The router acts as home gateway, network router, and access points for the
home network, providing Internet access for its users in the household. However, the default ISP
router commonly have limited networking features and capabilities, including network security. The
device also needs to be properly configured and managed in accordance to the usage and requirements
of the users inside the household.

\section{Goal}

The aim of this system development project is to develop home network security measure by using
available or affordable hardware.

\section{Objectives}

The objectives of this system development is to implement network security infrastructure in home
network environment by utilizing open-source network firewall hosted on personal computer.
Specifically, the objectives of this project are:

\begin{enumerate}

  % \item To develop home network security infrastructure by employing network firewall system hosted
  %   on a PC
  %
  % \item To examine the effectiveness of the developed system to secure home network
  %
  % \item To examine the effectiveness of the developed system to secure home network

  \item To study the problem which concern network security in home environment and existing
    solution to the identified problem.

  \item To study the advantage and disadvantage of existing solution to the identified network
    security concern in home environment.

  \item To propose solution which address the identified home environment network security concern
    and possible improvement from existing solution.

  \item To examine and test the functionality of the proposed system in addressing the identified
    network security concern in home environment.

\end{enumerate}

\section{Scope}

Project scope provides a clear understanding of what needs to be achieved and what is out of the
project's scope. It helps the project team and stakeholders to focus on the specific goals and
deliverables, avoiding unnecessary work or distractions. For this project, the scope are specified
below:

\begin{enumerate}

  \item The project will only cover the development of virtualised firewall and monitoring
    infrastructure in a personal computer. Any additional feature or issue that might be found in
    future development phase related to each software or subsystem will be referred to its
    respective repository.

    \item The exact hardware specification is not explicitly defined as users might choose for
      different specification depending on how much they are willing to spend. The project will only
      state the minimum hardware specification for optimal use based on the deployed subsystems
      requirement. Any performance-based metrics will not be compared.
  
\end{enumerate}

\section{Project Importance}

The development of his project will highlight the importance of a network security in a local
network in home environment. Network security measures should be taken into account in any network
where there is user inside the network to avoid unwanted loss and data theft.

\section{Report Organization}

This project report is consisted of five chapters. The first chapter is project Introduction which
contains problem background, project goal and objectives, project scope, and project importance. The
first chapter will help to further define the main aspects of the low-cost router and firewall
system development. The second chapter is Literature Review which consists of discussion and review
of existing system and technology that available to the end user. The gathered data then analyzed
and studied for the development of the system. The third chapter is project Methodology which
consists the discussion of the methodology chosen for the project development by justifying and
analyze the project requirement and objective. The fourth chapter is project Requirement Analysis
and System Design which covers the use case and initial design of the proposed system. The fifth and
last chapter is report Conclusion which discusses the achieved goals and objectives as result of the
report writing, as well as the plan for the continuation of the project development and suggestions
for future work. Project requirement use cases, activity diagram, and Gantt chart are included in
the Appendices.

\end{document}
