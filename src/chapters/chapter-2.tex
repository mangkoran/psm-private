\documentclass[../index.tex]{subfiles}

\begin{document}

\chapter{LITERATURE REVIEW}

This chapter will discuss several concepts and technologies which will be used in the proposed
system. There will be discussion regarding computer security concepts according to several
standards, followed by explanation of different types of firewalls and virtualization technologies,
brief explanation of Kubernetes system, and concluded with summary of the chapter.

% \section{Theoretical Basis}

\section{Computer Security}

According to \cite{Paulsen_Byers_2019}, the term computer security can be described as a collection
of actions and rules which ensure confidentiality, integrity, and availability of a certain
information system property. This includes the physical hardware, underlying software, embedded
firmware, and data that is transmitted, saved, and processed.

Derived from above definition, there are three key objectives which are the core of computer
security:

\subsection{Confidentiality}

The concept of Confidentiality covers the following two corresponding concepts:

\begin{description}

  \item[Data confidentiality] Assures that private or confidential information is not made
    accessible or shared to unauthorized users.

  \item[Privacy] Assures that personals manage or influence the collection and storage of
    information related to them and by whom and to whom that information may be shared.

\end{description}

\subsection{Integrity}

The concept of Integrity covers the following two corresponding concepts:

\begin{description}

  \item[Data integrity] Assures that information and programs are communicated through a particular
    and authorized method.

  \item[System integrity] Assures that a system serves its purpose in an unimpaired form, free from
    deliberate or unintended unauthorized manipulation of the system.

\end{description}

\subsection{Availability}

Assures that systems work properly and service is available authorized users.

% Another definition based on NIST standard FIPS 199 specifies the core of security goals for both
% information and information systems as confidentiality, integrity, and availability. FIPS 199
% establishes a classification of the aforementioned goals based on the condition and the
% understanding of security loss:
%
% \begin{description}
%
%   \item[Confidentiality] The term of confidentiality can be defined as the act of maintaining legal
%     restrictions on information access and sharing. This term includes the method for defending
%     privacy of an individual and private information.
%
%   \item[Integrity] The term integrity can be defined as the act of defending toward inappropriate
%     modification of information which result in information destruction, including maintaining
%     information non-repudiation and authenticity.
%
%   \item[Availability] The term availability can be defined as the act of ensuring information can be
%     accessed and used in reliable and promptly method.
%
% \end{description}
%
% While the aforementioned definition could be considered as well established, some computer security
% professionals further define additional concepts to present bigger picture. Two of the most commonly
% mentioned are follows:
%
% \begin{description}
%
%   \item[Authenticity] The characteristic of being legitimate and being able to be trusted and
%     verified. This characteristic will result in confidence in the legitimacy of transmission, a
%     message, or message author. This can be defined as verifying that users are the actual self that
%     they claim to be and that every message received by the system originated from a reliable
%     source.
%
%   \item[Accountability] The security objectives that create the condition for actions of an object
%     to be tracked down exclusively to that object. The concept includes deterrence, nonrepudiation,
%     intrusion detection and prevention, and fault isolation. In addition, the concept of
%     accountability also includes post-action restoration and judicial measures. As the fact that
%     fully secure system is not a realistic objective, user is required to track down a security
%     infringement in a system to a responsible team. Systems are required to safeguard the records of
%     their own activities to allow follow up forensic examination.
%
% \end{description}

\section{Threats and Attacks}

RFC 4949 defines four types of threat impacts and enumerates the attack types which lead to each
impact:

\subsection{Unauthorized disclosure}

RFC 4949 defines unauthorized disclosure as a situation which an individual acquires access to
certain data albeit the individual is not legitimate. There are four threat actions that could lead
to unauthorized disclosure:

\begin{description}

  \item[Exposure] An unauthorized individual is able to directly access to critical data.

  \item[Interception] An unauthorized individual interrupts critical data transmitted between
    legitimate origins and destinations.

  \item[Inference] A condition where an unauthorized individual is able to indirectly accesses
    critical data by understanding from the attributes or results of communications.

  \item[Intrusion] A condition where an unauthorized individual gains access to critical data by
    evading or beating the security guard of a system.

\end{description}

\subsection{Deception}

Deception is a condition which can lead in an authorized individual acquiring counterfeit data and
assuming it to be the actual data. There are three threat actions that could lead to deception:

\begin{description}

  \item[Masquerade] An unauthorized individual acquires access to a system or executes a harmful
    action by disguising as an authorized individual.

  \item[Falsification] A condition where an authorized individual is deceived by counterfeit data.

  \item[Repudiation] A condition where an individual tricks another by falsely denying obligation
    for a behavior.

\end{description}

\subsection{Disruption}

A condition can be categorised as a disruption when there is action that interrupts or prevents the
intended functionality of system services and utilities. There are three threat actions that could
lead to disruption:

\begin{description}

  \item[Incapacitation] Interrupts system operation by incapacitating a component of the system.

  \item[Corruption] Adversely mutates system operation by harmfully altering system data or
    functions.

  \item[Obstruction] A condition where system operation is being impeded by a threat which result in
    interruption of system services.

\end{description}

\subsection{Usurpation}

Usurpation is a condition where an unauthorized individual takeover the control system services or
functions. There are two threat actions that could lead to usurpation:

\begin{description}

  \item[Misappropriation] A condition where an individual presume unauthorized control of physical
    or logical resource of a certain system.

  \item[Misuse] A condition which may lead a system component to execute a function or behavior that
    is harmful to the security of the system.

\end{description}

\section{Firewall}

A firewall is a network security system which placed between local network and the Internet to
create a managed link and act as an external layer of security or boundary. The purpose of this
boundary is to safeguard the internal network from varying types of network-based threat and attack.
In addition, firewall offers a centralized control point which security and monitoring are able to
be established.

\subsection{Firewall Types}

A firewall has the functionality to supervise network traffic at multiple levels, from low-level
network packets, either single or as portion of a flow, onto the entire traffic within a transport
connection, up to examining the detailed information of application protocols. Firewall has
different functionalities and protection capabilities depending on its types. It can inspect more
than one protocol headers in every network packet, the content of every packet, or the arrangement
produced by packet sequences.

\begin{description}

  \item[Packet Filtering Firewall] A packet filtering firewall utilizes packet filters which inspect
    every packet that transmitted through the network. Afterward, based on the applied packet
    filters by the administrator, the firewall will allow or deny the packets to enter or exit the
    network. However, packet filtering firewall is known to be vulnerable to IP spoofing.

  \item[Circuit-Level Firewall] A circuit-level firewall can be either a dedicated appliance or a
    specific utility implemented by an application-level firewall for specific applications. With
    the same characteristics with an application-level firewall, this type of firewall has no
    ability to enable an end-to-end TCP connection. Circuit-level firewall works by establishing
    connection between two TCP port, one between the firewall and a TCP client on an internal
    network host and another one between the firewall and a TCP client on an external network host.
    After the two connections has been established, the gateway normally transmits segments of TCP
    beginning from one connection to the opposite 15 connection without inspecting the contents. As
    for the security aspects, the firewall will determine which of the connections are allowed.

  \item[Stateful Inspection Firewall] Stateful inspection firewall is also commonly known as third
    generation firewall technology. This type of firewall works by reinforcing the TCP traffic rules
    by utilizing the following techniques: traffic classification based on destination port of each
    traffic and packet recording of each interaction with internal connections. The previously
    mentioned technologies further increase the usability and helps broadening granularity of access
    control.

  \item[Application-Level Firewall] An application-level firewall, which also known as an
    application proxy, serves as a forwarder of traffic on application-level. This type of firewall
    provides layers of security mechanism on top of specified application, for example Telnet or FTP
    servers. In addition, the firewall also defines rules for HTTP/HTTPS connections which
    specifically built for every individual application to support identification and prevention of
    malicious traffic to the network

\end{description}

\subsection{Firewall Deployment}

Firewall system commonly deployed on a dedicated system which runs a specialized OS. The type of OS
which the firewall runs varies between firewall manufacturers.

Firewall utilities is also able to be implemented as a software module in an internet router or LAN
switch, or in a server. This section will explain several additional firewall basing considerations.

\begin{description}

  \item[Hardware Firewall] A hardware firewall can be described as an actual hardware appliance
    which provisions the traffic that went to and from a local network. To use this type of
    firewall, the internet cable is initially routed to the firewall appliance, as opposed to
    connecting it directly into the user device. The firewall is placed between the external network
    and the local network device, allowing the firewall to offer malware protection and an
    obstruction against intrusion

  \item[Software Firewall] While hardware firewall requires a dedicated hardware to run the firewall
    functionality, a software firewall could be installed on end-user device. Software firewall
    usually packaged as an application which run on the background of a host device rather than
    dedicated OS. Because of its characteristic, software firewall is more suited for personal use

  \item[Virtual Firewall] A virtual firewall is a software firewall implementation which is running
    on a virtual host managed by a hypervisor. As the firewall software is virtualised, the physical
    network interface is also required to be virtualised or shared ("passed- through"). One of the
    examples of physical device virtualization technology is Single Root I/O Virtualization (SR-IOV)
    specification.

\end{description}

\section{Virtualization}

Virtualization is computing technology which works by simulating an actual hardware function to
provide software-based IT services such as networks, storages, servers, and applications.
Virtualization technology works by using a software called hypervisors to establish a layer of
abstraction on top of physical hardware of the computer which enables the hardware components of one
computer (storage, processor, memory, etc.) to be shared or split into several virtual computers,
often referred to virtual machine (VMs). Every VM executes its own operating system (OS) and
operates like an independent computer albeit being run on only a fragment of the abstracted hardware

Available resources are divided as required from the actual system into several virtual
environments. This allows the users to interface with and execute computations inside the virtual
environment, which commonly referred as guest machine or virtual machine.

If a program or the user itself executes a command that demands additional resources from the actual
hardware while the user is running the virtual environment, the hypervisor sends the request to the
actual system and stores the modifications.

\subsection{Hypervisor Types}

Hypervisor is categorized under software layer application which manages VMs. Hypervisor functions
as communication port that enables VMs to communicate and interact with the actual physical
hardware, making sure that every VM has the ability to interact with the physical resources it
requires to execute. In addition, hypervisor also prevent the VMs interfering one another by
impinging on allocated memory space and compute cycles of other VM.

Based on how its software run on the host system, hypervisor can be divided into two types:

\begin{description}

  \item[Type 1 or Bare-Metal] Type 1 hypervisors also commonly called bare-metal hypervisors. This
    type of hypervisor communicates directly with the actual physical resources, substituting the
    conventional operation system completely. Type 1 hypervisors commonly found in virtual or cloud
    server environments.

  \item[Type 2] This type of hypervisors run as a regular software on top of an installed OS. This
    type of hypervisors mostly used for personal purposes, such as running alternative operating
    system. Type 2 hypervisors are known to have performance overhead which 18 caused by their need
    to access the host OS to access and manage the actual hardware resources.

\end{description}

\subsection{Virtualization Types}

While the most common virtualization type is local desktop virtualization, it also exists other
types of virtualization technology which are developed with specific use cases and purposes.

\begin{description}

  \item[Desktop Virtualization] Desktop virtualization allows its users to run multiple desktop
    operating systems with each in its individual VM on the same computer system. Desktop
    virtualization can be categorized into two types: local desktop virtualization and Virtual
    Desktop Infrastructure (VDI).

    Local desktop virtualization runs the hypervisor software on a local computer. This allows the
    user to have more than one additional OS on their system and interchange to each OS as necessary
    without being required to modify anything about the primary OS. On the other hand, Virtual
    Desktop Infrastructure allows running several desktops in VMs on a single centralized server and
    shares them to users who access it on the allowed devices. With this approach, VDI enables a
    company to gain its users the ability to access variety of OSes regardless of the device,
    without being required to install the OSes on any device.

  \item[Network Virtualization] The term network virtualization refers to a software which help
    create a "standpoint" of the network which allows an administrator to control the managed
    network from a single control point. Network virtualization works by abstracting physical
    hardware and functions (e.g., routers, switches, links, etc.) into software which managed by a
    hypervisor. Network virtualization allows the network administrator to modify and control the
    abstracted elements without being required to physically touch the actual physical hardware,
    which significantly streamlines the management of the network.

    There are several types of network virtualization. The first type is software- defined
    networking (SDN) which works by virtualizing appliance which manages the routing of network
    traffic, which also commonly recognized as the “control plane”. Another type of network
    virtualization is network function virtualization (NFV) which works by virtualizing more than
    one hardware system which offers a particular network function (e.g., a traffic analyzer, load
    balancer, or firewall), allowing the virtualized appliances to become easier to set up, manage,
    and monitor

  \item[Storage Virtualization] Storage virtualization allows the entire of storage appliances which
    exist in a network, regardless whether the storage is installed on a particular system or
    separate dedicated storage units, to be accessed and controlled as one big storage device. In
    particular, storage virtualization merges all blocks of storage into one big pool which allows
    them to be allocated to any VM on the particular network as required. Storage virtualization
    allows simpler storage monitoring for VMs and optimize utilization of every existing storage on
    the particular network.

  \item[Data Virtualization] Modern organizations store data from various applications, using
    various file formats, in various locations, varying from the cloud to on-premise hardware and
    software solutions. Data virtualization allows any application to access all of the stored data,
    regardless of its source, format, or location.

    Data virtualization utilities establish a software layer between the applications accessing the
    data and the systems where the data is stored. The layer interprets data request of a certain
    application or query as required and provides results that might be retrieved from several
    systems. This kind of virtualization enables the ability to break down data depots in case of
    other kinds of data integration are not affordable, desirable, or feasible

  \item[Application Virtualization] Application virtualization runs application software in a
    virtualized environment aside from directly install the application on the OS of the user. By
    using this method, it provides different approach from full-fledged desktop virtualization as
    the virtual environment only runs the defined application, while the end user OS on their device
    runs without being virtualized. The following are the categories of application virtualization:
    Local appliation virtualization, application streaming, and server-based application
    virtualization.

    In local application virtualization, The whole application will be run on the end device
    although it will be run in a contained environment in contrast to the actual hardware. In recent
    years, this technology has been evolving to what commonly known as "containerization"

    Meanwhile, application streaming allow the application to be hosted on a server which shares a
    portion of the software to be executed on the user device when required.

    In server-based application virtualization, user will use the software that act as the
    application’s user interface to access the actual application which runs fully on the server.

  \item[Data Centre Virtualisation] Data center virtualization works by abstracting the entire
    hardware of the data center into software, allowing an administrator to segment a single actual
    hardware data center into many virtual data centers to be used by various users.

    Every user of the virtualized data center is able to interact with its own infrastructure as a
    service (IaaS), which is being run on the same actual system. Virtual data centers allow a
    smooth transition into cloud-based computing, enabling an organization to rapidly configure
    full-fledged data center environment without being required to purchase infrastructure hardware.

  \item[CPU Virtualization] Central processing unit (CPU) virtualization is one of the core
    technologies that enables hypervisors, virtual machines, and operating systems development. The
    technology allows one CPU to be partitioned into multiple virtual CPUs for use by several VMs.

  \item[GPU Virtualization] A graphical processing unit (GPU) is a specialized multi core processor
    that optimizes the performance of the computer system by addressing heavy-load graphic or
    arithmetic processing. This type of virtualization allows several VMs to use some or even the
    entirety of the processing power of a single or multiple GPUs for improved video encoding and
    decoding, artificial intelligence (AI) inferencing, and other kinds of heavy computational
    tasks.

  \item[Linux Virtualization] Since the release of version 2.6.20, users using Linux are able to use
    Linux based hypervisor, widely known as the kernel-based virtual machine (KVM). The Linux KVM is
    supported by Intel and AMD-based virtualization processor extensions that enables its users to
    virtualize VMs which is based on x86 from within a Linux host OS.

  \item[Cloud Virtualisation] The cloud computing system model heavily relies on virtualization
    technology. The usage of virtualization technology for its hardware such as storage, processor,
    memory and other server components allow cloud computing providers to offer a wide variety of
    utilities to their customers.

\end{description}

\section{Current System Analysis}

In the present days, consumer who are subscribed to internet service provided by an Internet Service
Provider (ISP) mostly provided with a router that capable to transmit its connection over Wi-Fi to
local device. This kind of router, often called home router, provide basic functionality as a
network router which allows the local user to access the Internet. However, most kinds of home
router have minimal network security capability.

In the other hand, enterprise firewall solution addresses the requirement for proper and extensive
network security for its customer’s network. Several specialized functionalities such as firewall
rule, traffic shaping, and intrusion detection are mostly only available in enterprise firewall
appliance. However, this kind of firewall has high price. This makes implementing proper network
security in home network environment considerably hard.

To study and analyze the capabilities and shortcomings of currently existing firewall systems, two
devices are chosen to be inspected: one home router device and one enterprise-grade firewall device.
The first device chosen is Kaon AR2140, the router that provided by Maxis for its Maxis Fiber
subscriber. The Kaon AR2140 main features are support for Wi-Fi 6 and EasyMesh certification.

For the enterprise firewall device, Deciso DEC675 is chosen to be compared. The DEC675 is a desktop
security appliance sold by Deciso, the parent company of OPNsense. While it is commercialized, it
utilizes the open-source OPNsense firewall operating system at its core. The only addition is
limited Business Edition license, which includes access to commercial firmware repository and
professional support. Since it uses OPNsense, it offers the same security and networking features
that offered by OPNsense.

\section{Existing System Analysis}

In the present days, consumer who are subscribed to internet service provided by an Internet Service
Provider (ISP) mostly provided with a router that capable to transmit its connection over Wi-Fi to
local device. This kind of router, often called home router, provide basic functionality as a
network router which allows the local user to access the Internet. However, most kinds of home
router have minimal network security capability.

In the other hand, enterprise firewall solution addresses the requirement for proper and extensive
network security for its customer’s network. Several specialized functionalities such as firewall
rule, traffic shaping, and intrusion detection are mostly only available in enterprise firewall
appliance. However, this kind of firewall has high price. This makes implementing proper network
security in home network environment considerably hard.

To study and analyze the capabilities and shortcomings of currently existing firewall systems, two
devices are chosen to be inspected: one home router device and one enterprise-grade firewall device.
The first device chosen is Kaon AR2140, the router that provided by Maxis for its Maxis Fiber
subscriber. The Kaon AR2140 main features are support for Wi-Fi 6 and EasyMesh certification.

For the enterprise firewall device, Deciso DEC675 is chosen to be compared. The DEC675 is a desktop
security appliance sold by Deciso, the parent company of OPNsense. While it is commercialized, it
utilizes the open-source OPNsense firewall operating system at its core. The only addition is
limited Business Edition license, which includes access to commercial firmware repository and
professional support. Since it uses OPNsense, it offers the same security and networking features
that offered by OPNsense.

\begin{table}[h!]
  \begin{tabularx}{\textwidth}{|X|c|c|c|} 
    \hline
    \multicolumn{1}{|c|}{Feature} & \multicolumn{1}{c|}{AR2140} & \multicolumn{1}{c|}{DEC675} & \multicolumn{1}{c|}{Proposed System} \\
    \hline
    % Brand                        & Kaon & Deciso & Lenovo \\ 
    Firewall Rule                & - & \checkmark & \checkmark \\ 
    IDS/IPS                      & - & \checkmark & \checkmark \\ 
    DNS Blocklist                & - & \checkmark & \checkmark \\ 
    \hline
  \end{tabularx}
  \caption{Comparison of Existing Systems}
  \label{table:existing_system_comparison}
\end{table}

\Cref{table:existing_system_comparison} shows the comparison between Kaon AR2140, Deciso DEC675, and
the proposed system.

\section{Technology Used}

This section will discuss the technology used to develop the proposed system. This discussion will
include both the tools used to code and programs that will be deployed in the proposed system will
be included.

% \subsection{Deployment Infrastructure}

\subsection{Virtualisation Platform}

Virtualization allows users to create multiple virtual machines running different operating system
(OS) abstracted from its hardware resources. As the use case of these virtual machines may vary from
one deployment to the other, there are several virtualization environment solutions available for
users.

VirtualBox is one of the most widely used virtualization platform. VirtualBox can be categorized as
type 2 hypervisor originally developed by Innotek GmbH, now acquired by Oracle Corporation. As a
type 2 hypervisor, VirtualBox can be installed on an existing OS installation without the need to
format and initialize from zero. It allows the creation and virtualization of any x86-compatible OS
on top of several supported OS, which are: OpenSolaris, Solaris, Linux, macOS, and Microsoft
Windows. VirtualBox is a free and open-source software, released under GNU General Public License,
although it has proprietary optional Extension Pack available.

VMware Workstation is another popular type 2 hypervisor virtualization platform. VMware Workstation
is developed by VMware, initially released in 1999. It has most of the features that VirtualBox has,
with several additional capabilities such as allow higher VM video memory, support more recent
version of DirectX and OpenGL, and have out-of-the-box support for USB peripherals. However, as
VMware Workstation is a trialware, user is required to purchase a license in order to use it beyond
its trial limit.

Another option is Proxmox Virtual Environment (Proxmox VE). Proxmox VE is developed by Proxmox
Server Solutions GmbH, with its first release made it to public on 15 April 2008. Proxmox VE can be
categorized as type 1 hypervisor, as it leverages the usage of KVM or QEMU and does not require any
base OS for its deployment. Being type 1 or native hypervisor, Proxmox VE has lower overhead
compared to type 2 or software hypervisor as it runs on bare metal, allowing its hypervisor to
communicate directly with the underlying hardware components. Additionally, Proxmox VE also support
Linux Container (LXC) which utilize OS- level virtualization. As the result, LXC has minimal
overhead compared to other virtualization method. Instead of creating an entire virtual machine, LXC
achieves its virtualization by utilizing virtual environment which has separate network and process
space. Proxmox VE is free and open-source software, released under GNU Affero General Public License
version 3 with optional paid subscription to gain access to their Enterprise repository and
professional support team.

VMware ESXi is also an alternative option for virtualization platform. VMware ESXi, as stated in its
name, is developed by VMware as type 1 hypervisor capable for deploying and managing virtual
machines. It was first released on 23 March 2001, with current stable version is 7.0 at the time of
writing. VMware ESXi benefits from being a type 1 or native hypervisor, such as lower process
overhead and smaller footprint. VMware ESXi is released under proprietary license, meaning that in
order to use the program fully without being limited to free trial, user must purchase vSphere
subscription which cost around USD 580 for its most affordable tier at the time of writing.

Based on several justifications that follow, Proxmox VE is chosen as the virtualization platform for
the system development project. Proxmox VE is a free and open-source software, meaning that it can
be fully utilized without the need to purchase the software or even subscription. Furthermore,
Proxmox VE is a type 1 or native hypervisor, which has less overhead and more controls over type 2
hypervisor. It also supports LXC which will help Pi-hole deployment on the system

\subsection{Firewall System}

In order to establish a proper network security for home network, a suitable and capable firewall
system must be used. Several firewall operating systems are available publicly, each of them has
different feature set and optimized use case.

OpenWrt is a free and open-source embedded operating system optimized for embedded device for
network traffic routing. OpenWrt is developed based on Linux, BusyBox, musl, and util-linux. Every
part of the OS has gone through optimization to be as small as possible to fit into home routers
which have limited amount of memory and storage. OpenWrt has common networking features such as but
not limited to packet routing, NAT, DHCP, and DNS. Being a free and open-source project allows the
development of many additional extension which add various extra functionality to OpenWrt, such as
but not limited to mesh networking, IPS, active queue management, and IP tunneling. Furthermore, it
requires zero cost to deploy and utilize OpenWrt as it is provided through GNU General Public
License.

Another option is pfSense. pfSense is a firewall and router operating system distribution which is
developed on top of FreeBSD. pfSense has different main development objective than OpenWrt. Rather
than developing minimalist networking OS solution for embedded system, pfSense is developed with
network firewall as its primary core. Because of that, it has more network security features built
into the core OS compared to OpenWrt, such as but not limited to WireGuard, traffic shaping, IPsec,
and L2TP VPN. Initially, pfSense is developed as a the m0n0wall project’s fork. Since then, Netgate
continue its development by dividing into two producs: pfSense Community Edition (pfSense CE) and
pfSense Plus (formerly pfSense Factory Edition). pfSense CE is released under open-source Apache
License 2, while pfSense Plus is released as proprietary product, mostly come pre-installed in
Netgate networking appliances. Although Netgate still continue the development of pfSense CE, it is
updated less often than pfSense Plus, as Netgate shifted their focus to more profitable product.

As a response to several concern of the future direction of pfSense open-source project, a fork of
the project is created under the name OPNsense. The OPNsense project is started as fork of pfSense
project after the original m0n0wall project was announced to be discontinued. OPNsense is currently
developed with Deciso B.V. as its main contributor and supervisor. Most of OPNsense features are
similar to pfSense, with more frequent release and transparent development roadmap. OPNsense is
released under BSD 2-Clause “Simplified” license. Because of that, all of OPNsense source code are
available to public, including its build tools and utilities. In contrast, pfSense only released the
code for its “core” component only, with its build tools remain private.

With regards to several considerations, OPNsense will be used as the firewall system for the
project’s system. One of the main justifications is its software licensing and capability. OPNsense
has clear and straightforward BSD 2-Clause license, which allows both personal and enterprise
deployment at no cost. Moreover, OPNsense development is primarily focused on firewall and network
security capability, in contrast to OpenWrt which is developed for embedded system networking. This
makes OPNsense more suitable platform to build home network security infrastructure.

\subsection{DNS Server}

Although there are several options for DNS server that are available to use in OPNsense firewall
such as BIND and Unbound, both these DNS servers have no fine control over blocklist source and
custom whitelist and blacklist rule. 

Pi-hole is a DNS server and sinkhole that act as a network-level ad blocker and filtering system for
internet traffic. It is designed to run on a Raspberry Pi or other devices like Linux servers and
can be used to block advertisements, tracking domains, and malicious websites at the network level,
thereby improving internet browsing experience and privacy.

\subsection{Monitoring Stack}

A monitoring system is essential for a system as it provides real-time visibility and insights into
the performance, availability, and health of various systems, networks, applications, and services.

\section{Summary}

This chapter discussed the definition of computer security, followed by comparison of currently
existing solutions, along with technology used in the system. Currently existing solution may
unsuitable for use by small business that requires a more capable firewall system in an affordable
price point. This project tries to develop a low-cost firewall appliance by implementing virtualized
firewall OS on a consumer PC. The usage of consumer PC allows the user to upgrade the system in
later time if their requirement becoming more demanding.

\end{document}
