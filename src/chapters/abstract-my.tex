\documentclass[../index.tex]{subfiles}

\begin{document}

\clearpage
\chapter*{Abstrak}
\addcontentsline{toc}{chapter}{ABSTRAK}

Memandangkan aplikasi teknologi maklumat dan komunikasi telah meningkat secara eksponen sejak
beberapa tahun kebelakangan ini, peranti pintar dan pintar mula muncul dalam persekitaran rumah.
Perkembangan teknologi wayarles mencipta kebebasan yang besar untuk dan lebih banyak mobiliti
pengguna dengan peranti mudah alih. Ini juga meningkatkan penggunaan peranti mudah alih di rumah,
yang membawa kepada kelahiran rangkaian wayarles di rumah. Walau bagaimanapun, penggunaan meluas ICT
dalam isi rumah mencipta risiko dan ancaman keselamatan baharu terhadap aset persendirian pengguna
di dalam rangkaian rumah. Ia menjadi lebih menyusahkan kerana kebanyakan individu dalam persekitaran
rumah kurang mahir dan berkemampuan untuk bertindak balas terhadap serangan penceroboh dan
penggodam. Kuda Trojan, penafian perkhidmatan, dan menghidu paket adalah beberapa daripada banyak
serangan siber yang boleh menjejaskan komputer dan peranti rumah yang tidak dilindungi. Antara
langkah yang boleh dilakukan ialah menggunakan firewall rangkaian untuk mewujudkan keselamatan
rangkaian rumah. Tembok api rangkaian menyediakan lebih banyak ciri keselamatan dan kawalan
rangkaian ke atas tembok api berasaskan hos. Projek ini berhasrat untuk membangunkan bukti konsep
dalam mewujudkan keselamatan rangkaian rumah dengan menggunakan sistem pengendalian firewall gred
perusahaan yang digunakan pada komputer peribadi. Keperluan untuk sistem tembok api akan disemak
dalam fasa analisis keperluan. Selepas keperluan sistem telah diperiksa, projek akan membincangkan
dan menentukan reka bentuk sistem. Selepas itu, pelaksanaan sistem awal berdasarkan reka bentuk yang
ditentukan akan menyusul. Sistem akan diuji kemudian mengikut keperluan dan objektif sistem sebelum
ia boleh digunakan dalam persekitaran sebenar. Sistem ini dijangka dapat menyediakan infrastruktur
keselamatan rangkaian yang sesuai untuk persekitaran rangkaian rumah

\clearpage

\end{document}
