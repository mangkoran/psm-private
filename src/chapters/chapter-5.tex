\documentclass[../index.tex]{subfiles}

\begin{document}

\chapter{IMPLEMENTATION AND TESTING}

Impelementation and Testing chapter explains the system impelementation which includes the
configuration lines of the system that were carried out as continuation from the design and analysis
phase. This phase is exceedingly critical for the project as ... achieve the system requirements.

Both black-box and white-box testing will be employed to ensure the functionality and system
internals are operating correctly to meet the system requirements.

\section{Main System Functionality Code}

In this section, the code of main functionality of the system will be attached and followed by the
explanation of the respective code. The complete details of the code design will be documented in
Appendix A.

\subsection{OPNSense}

The previous are the Ansible playbook and task to bootstrap a KVM virtual machine that will host
OPNSense firewall. The Ansible playbook will firstly check whether a KVM virtual machine with the
name opnsense1 is already existed. If there is no such virtual machine, the Ansible playbook will
create a new KVM virtual machine. As with any KVM virtual machine, manual installation procedure is
required for the initial OPNSense setup by booting the virtual machine with OPNSense installation
image ISO file.

\subsection{Pi-hole}

The previous are the Ansible playbook and task to bootstrap an Ubuntu LXC container that will host
Pi-hole DNS server. The Ansible playbook will firstly check whether an LXC container with the name
pihole1 is already existed. If there is no such container, the Ansible playbook will create a new
Ubuntu container. After a new container is created, the Ansible playbook will check whether pihole
program is already installed. If there is no pihole program installed, the Ansible playbook will
proceed to install pihole using officially provided install script. Before the installation, the
Ansible playbook will ensure the required dependencies are installed. Followed by that, the Ansible
playbook will copy the unattended setup configuration file from the control node to the pihole1
container. After both the dependencies and setup configuration exist, the Ansible playbook will
install pihole using the install script.

In addition to the Pi-hole DNS server, the Ansible playbook will configure Rsyslog so that any DNS
action log produced by Pi-hole will be forwarded to the monitoring1 container for log processing.

\subsection{Monitoring}

The previous are the Ansible playbook and task to bootstrap an Ubuntu LXC container that whill host
the monitoring stack of the system. The monitoring stack consists of Vector, Prometheus, Loki, and
Grafana. The ansible playbook will firstly check whether an LXC container with the name monitoring1
is already existed. If there is no such container, the Ansible playbook will create a new Ubuntu
container.

% https://blog.dmcindoe.dev/posts/2021-07-31/automating-proxmox-with-ansible/

\end{document}
